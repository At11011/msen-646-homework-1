\input{./src/main.sty}
% Additional SI unit for Fahrenheit
\DeclareSIUnit\fahrenheit{\degree F}

\begin{document}

% Include title page
\input{./src/titlepage.tex}

\section{Problem 1}

Please find the magnitudes missing for each electrical circuit expressed.

\begin{enumerate}[a]
    \item $I_1, I_2, I_\text{total}$ (across the resistance), $V_2, R_\text{total}$?

        \begin{figure}[h]
    \centering
    \includegraphics[width=0.6\textwidth]{./assets/fig_1.png}
\end{figure}

\begin{itemize}
    \item $E_1 = \SI{8}{\volt}$
    \item $E_2 = \SI{4}{\volt}$
    \item $R_1 = \SI{2}{\ohm}$
    \item $R_2 = \SI{4}{\ohm}$
\end{itemize}


\pagebreak

    \item $R_\text{total}, I_1, I_2, I_3, V_1, V_2, V_3$?

        \begin{figure}[h]
    \centering
    \includegraphics[width=0.6\textwidth]{./assets/fig_2.png}
\end{figure}

\begin{itemize}
    \item $E_\text{total} = \SI{5}{\volt}$
    \item $R_2 = \SI{20}{\ohm}$
    \item $R_3 = \SI{30}{\ohm}$
\end{itemize}

\pagebreak

    \item $I_\text{total}, I_1, I_2, I_3, R_\text{total}$?

        \begin{figure}[h]
    \centering
    \includegraphics[width=0.6\textwidth]{./assets/fig_3.png}
\end{figure}

\begin{itemize}
    \item $E_1 = \SI{10}{\volt}$
    \item $E_2 = \SI{20}{\volt}$
    \item $R_1 = \SI{5}{\ohm}$
    \item $R_2 = \SI{10}{\ohm}$
    \item $R_3 = \SI{1}{\ohm}$
    \item $I_4 = \SI{10}{\milli\ampere}$
\end{itemize}

\pagebreak

\item $R_\text{total}, V_\text{total}, I_\text{total}, I_1, I_2, I_3$?

        \begin{figure}[h]
    \centering
    \includegraphics[width=0.6\textwidth]{./assets/fig_4.png}
\end{figure}

\begin{itemize}
    \item $E_1 = \SI{20}{\volt}$
    \item $R_1 = \SI{10}{\ohm}$
    \item $R_2 = \SI{20}{\ohm}$
    \item $R_3 = \SI{5}{\ohm}$
\end{itemize}

\pagebreak

\item Perform the following reference conversions

    \begin{enumerate}
        \item Convert \SI{150}{\milli\volt} SCE to CSE
        \item Convert \SI{-150}{\milli\volt} SCE to CSE
        \item Convert \SI{-1.0}{\volt} CSE to the equivalent Ag/AgCl reference electrode
        \item Convert \SI{-0.75}{\volt} to the equivalent Zn reference electrode
        \item Covert \SI{0.00}{\volt} Zn to the equivalent CSE
        \item Convert \SI{-0.05}{V} Zn to the equivalent Ag/AgCl reference electrode
    \end{enumerate}

\pagebreak

    \item In a laboratory test, the corrosion current density of a steel AISI
        1010 sample is \SI{50}{\milli\ampere\per\centi\meter\squared}, can you 
        calculate the corrosion rate for the alloy?

    Assume the alloy contains the following composition:

    \begin{figure}[h]
        \centering
        \includegraphics[width=0.6\textwidth]{./assets/fig_5.png}
    \end{figure}

    NOTE: Be sure what elements will be anodic or will dissolve in order to 
    make the final calculation of the alloy.

\item If an iron pipeline is immersed in a soil that is $\mathrm{pH}=8$, what 
    electrode potential must be maintained to prevent corrosion of the bare 
    metal versus the \ch{Cu/CuSO4} electrode? If you have a steel structure 
    that is buried in soil and is located thermodynamically at point A, what
    is the best strategy to have a "corrosion healthy structure"? Name two
    strategies.

    \begin{figure}[h]
        \centering
        \includegraphics[width=0.6\textwidth]{./assets/fig_6.png}
    \end{figure}

\end{enumerate}

\end{document}

% Additional SI unit for Fahrenheit
\DeclareSIUnit\fahrenheit{\degree F}

\begin{document}

% Include title page
\input{./src/titlepage.tex}

\section{Problem 1}

Please find the magnitudes missing for each electrical circuit expressed.

\begin{enumerate}[a]
    \item $I_1, I_2, I_\text{total}$ (across the resistance), $V_2, R_\text{total}$?

        \begin{figure}[h]
    \centering
    \includegraphics[width=0.6\textwidth]{./assets/fig_1.png}
\end{figure}

\begin{itemize}
    \item $E_1 = \SI{8}{\volt}$
    \item $E_2 = \SI{4}{\volt}$
    \item $R_1 = \SI{2}{\ohm}$
    \item $R_2 = \SI{4}{\ohm}$
\end{itemize}


\pagebreak

    \item $R_\text{total}, I_1, I_2, I_3, V_1, V_2, V_3$?

        \begin{figure}[h]
    \centering
    \includegraphics[width=0.6\textwidth]{./assets/fig_2.png}
\end{figure}

\begin{itemize}
    \item $E_\text{total} = \SI{5}{\volt}$
    \item $R_2 = \SI{20}{\ohm}$
    \item $R_3 = \SI{30}{\ohm}$
\end{itemize}

\pagebreak

    \item $I_\text{total}, I_1, I_2, I_3, R_\text{total}$?

        \begin{figure}[h]
    \centering
    \includegraphics[width=0.6\textwidth]{./assets/fig_3.png}
\end{figure}

\begin{itemize}
    \item $E_1 = \SI{10}{\volt}$
    \item $E_2 = \SI{20}{\volt}$
    \item $R_1 = \SI{5}{\ohm}$
    \item $R_2 = \SI{10}{\ohm}$
    \item $R_3 = \SI{1}{\ohm}$
    \item $I_4 = \SI{10}{\milli\ampere}$
\end{itemize}

\pagebreak

\item $R_\text{total}, V_\text{total}, I_\text{total}, I_1, I_2, I_3$?

        \begin{figure}[h]
    \centering
    \includegraphics[width=0.6\textwidth]{./assets/fig_4.png}
\end{figure}

\begin{itemize}
    \item $E_1 = \SI{20}{\volt}$
    \item $R_1 = \SI{10}{\ohm}$
    \item $R_2 = \SI{20}{\ohm}$
    \item $R_3 = \SI{5}{\ohm}$
\end{itemize}

\pagebreak

\item Perform the following reference conversions

    \begin{enumerate}
        \item Convert \SI{150}{\milli\volt} SCE to CSE
        \item Convert \SI{-150}{\milli\volt} SCE to CSE
        \item Convert \SI{-1.0}{\volt} CSE to the equivalent Ag/AgCl reference electrode
        \item Convert \SI{-0.75}{\volt} to the equivalent Zn reference electrode
        \item Covert \SI{0.00}{\volt} Zn to the equivalent CSE
        \item Convert \SI{-0.05}{V} Zn to the equivalent Ag/AgCl reference electrode
    \end{enumerate}

\pagebreak

    \item In a laboratory test, the corrosion current density of a steel AISI
        1010 sample is \SI{50}{\milli\ampere\per\centi\meter\squared}, can you 
        calculate the corrosion rate for the alloy?

    Assume the alloy contains the following composition:

    \begin{figure}[h]
        \centering
        \includegraphics[width=0.6\textwidth]{./assets/fig_5.png}
    \end{figure}

    NOTE: Be sure what elements will be anodic or will dissolve in order to 
    make the final calculation of the alloy.

\item If an iron pipeline is immersed in a soil that is $\mathrm{pH}=8$, what 
    electrode potential must be maintained to prevent corrosion of the bare 
    metal versus the \ch{Cu/CuSO4} electrode? If you have a steel structure 
    that is buried in soil and is located thermodynamically at point A, what
    is the best strategy to have a "corrosion healthy structure"? Name two
    strategies.

    \begin{figure}[h]
        \centering
        \includegraphics[width=0.6\textwidth]{./assets/fig_6.png}
    \end{figure}

\end{enumerate}

\end{document}

% Additional SI unit for Fahrenheit
\DeclareSIUnit\fahrenheit{\degree F}

\begin{document}

% Include title page
\input{./src/titlepage.tex}

\section{Problem 1}

Please find the magnitudes missing for each electrical circuit expressed.

\begin{enumerate}[a]
    \item $I_1, I_2, I_\text{total}$ (across the resistance), $V_2, R_\text{total}$?

        \begin{figure}[h]
    \centering
    \includegraphics[width=0.6\textwidth]{./assets/fig_1.png}
\end{figure}

\begin{itemize}
    \item $E_1 = \SI{8}{\volt}$
    \item $E_2 = \SI{4}{\volt}$
    \item $R_1 = \SI{2}{\ohm}$
    \item $R_2 = \SI{4}{\ohm}$
\end{itemize}


\pagebreak

    \item $R_\text{total}, I_1, I_2, I_3, V_1, V_2, V_3$?

        \begin{figure}[h]
    \centering
    \includegraphics[width=0.6\textwidth]{./assets/fig_2.png}
\end{figure}

\begin{itemize}
    \item $E_\text{total} = \SI{5}{\volt}$
    \item $R_2 = \SI{20}{\ohm}$
    \item $R_3 = \SI{30}{\ohm}$
\end{itemize}

\pagebreak

    \item $I_\text{total}, I_1, I_2, I_3, R_\text{total}$?

        \begin{figure}[h]
    \centering
    \includegraphics[width=0.6\textwidth]{./assets/fig_3.png}
\end{figure}

\begin{itemize}
    \item $E_1 = \SI{10}{\volt}$
    \item $E_2 = \SI{20}{\volt}$
    \item $R_1 = \SI{5}{\ohm}$
    \item $R_2 = \SI{10}{\ohm}$
    \item $R_3 = \SI{1}{\ohm}$
    \item $I_4 = \SI{10}{\milli\ampere}$
\end{itemize}

\pagebreak

\item $R_\text{total}, V_\text{total}, I_\text{total}, I_1, I_2, I_3$?

        \begin{figure}[h]
    \centering
    \includegraphics[width=0.6\textwidth]{./assets/fig_4.png}
\end{figure}

\begin{itemize}
    \item $E_1 = \SI{20}{\volt}$
    \item $R_1 = \SI{10}{\ohm}$
    \item $R_2 = \SI{20}{\ohm}$
    \item $R_3 = \SI{5}{\ohm}$
\end{itemize}

\pagebreak

\item Perform the following reference conversions

    \begin{enumerate}
        \item Convert \SI{150}{\milli\volt} SCE to CSE
        \item Convert \SI{-150}{\milli\volt} SCE to CSE
        \item Convert \SI{-1.0}{\volt} CSE to the equivalent Ag/AgCl reference electrode
        \item Convert \SI{-0.75}{\volt} to the equivalent Zn reference electrode
        \item Covert \SI{0.00}{\volt} Zn to the equivalent CSE
        \item Convert \SI{-0.05}{V} Zn to the equivalent Ag/AgCl reference electrode
    \end{enumerate}

\pagebreak

    \item In a laboratory test, the corrosion current density of a steel AISI
        1010 sample is \SI{50}{\milli\ampere\per\centi\meter\squared}, can you 
        calculate the corrosion rate for the alloy?

    Assume the alloy contains the following composition:

    \begin{figure}[h]
        \centering
        \includegraphics[width=0.6\textwidth]{./assets/fig_5.png}
    \end{figure}

    NOTE: Be sure what elements will be anodic or will dissolve in order to 
    make the final calculation of the alloy.

\item If an iron pipeline is immersed in a soil that is $\mathrm{pH}=8$, what 
    electrode potential must be maintained to prevent corrosion of the bare 
    metal versus the \ch{Cu/CuSO4} electrode? If you have a steel structure 
    that is buried in soil and is located thermodynamically at point A, what
    is the best strategy to have a "corrosion healthy structure"? Name two
    strategies.

    \begin{figure}[h]
        \centering
        \includegraphics[width=0.6\textwidth]{./assets/fig_6.png}
    \end{figure}

\end{enumerate}

\end{document}

% Additional SI unit for Fahrenheit
\DeclareSIUnit\fahrenheit{\degree F}

\begin{document}

% Include title page
\input{./src/titlepage.tex}

\section{Problem 1}

Please find the magnitudes missing for each electrical circuit expressed.

\begin{enumerate}[a]
    \item $I_1, I_2, I_\text{total}$ (across the resistance), $V_2, R_\text{total}$?

        \begin{figure}[h]
    \centering
    \includegraphics[width=0.6\textwidth]{./assets/fig_1.png}
\end{figure}

\begin{itemize}
    \item $E_1 = \SI{8}{\volt}$
    \item $E_2 = \SI{4}{\volt}$
    \item $R_1 = \SI{2}{\ohm}$
    \item $R_2 = \SI{4}{\ohm}$
\end{itemize}


\pagebreak

    \item $R_\text{total}, I_1, I_2, I_3, V_1, V_2, V_3$?

        \begin{figure}[h]
    \centering
    \includegraphics[width=0.6\textwidth]{./assets/fig_2.png}
\end{figure}

\begin{itemize}
    \item $E_\text{total} = \SI{5}{\volt}$
    \item $R_2 = \SI{20}{\ohm}$
    \item $R_3 = \SI{30}{\ohm}$
\end{itemize}

\pagebreak

    \item $I_\text{total}, I_1, I_2, I_3, R_\text{total}$?

        \begin{figure}[h]
    \centering
    \includegraphics[width=0.6\textwidth]{./assets/fig_3.png}
\end{figure}

\begin{itemize}
    \item $E_1 = \SI{10}{\volt}$
    \item $E_2 = \SI{20}{\volt}$
    \item $R_1 = \SI{5}{\ohm}$
    \item $R_2 = \SI{10}{\ohm}$
    \item $R_3 = \SI{1}{\ohm}$
    \item $I_4 = \SI{10}{\milli\ampere}$
\end{itemize}

\pagebreak

\item $R_\text{total}, V_\text{total}, I_\text{total}, I_1, I_2, I_3$?

        \begin{figure}[h]
    \centering
    \includegraphics[width=0.6\textwidth]{./assets/fig_4.png}
\end{figure}

\begin{itemize}
    \item $E_1 = \SI{20}{\volt}$
    \item $R_1 = \SI{10}{\ohm}$
    \item $R_2 = \SI{20}{\ohm}$
    \item $R_3 = \SI{5}{\ohm}$
\end{itemize}

\pagebreak

\item Perform the following reference conversions

    \begin{enumerate}
        \item Convert \SI{150}{\milli\volt} SCE to CSE
        \item Convert \SI{-150}{\milli\volt} SCE to CSE
        \item Convert \SI{-1.0}{\volt} CSE to the equivalent Ag/AgCl reference electrode
        \item Convert \SI{-0.75}{\volt} to the equivalent Zn reference electrode
        \item Covert \SI{0.00}{\volt} Zn to the equivalent CSE
        \item Convert \SI{-0.05}{V} Zn to the equivalent Ag/AgCl reference electrode
    \end{enumerate}

\pagebreak

    \item In a laboratory test, the corrosion current density of a steel AISI
        1010 sample is \SI{50}{\milli\ampere\per\centi\meter\squared}, can you 
        calculate the corrosion rate for the alloy?

    Assume the alloy contains the following composition:

    \begin{figure}[h]
        \centering
        \includegraphics[width=0.6\textwidth]{./assets/fig_5.png}
    \end{figure}

    NOTE: Be sure what elements will be anodic or will dissolve in order to 
    make the final calculation of the alloy.

\item If an iron pipeline is immersed in a soil that is $\mathrm{pH}=8$, what 
    electrode potential must be maintained to prevent corrosion of the bare 
    metal versus the \ch{Cu/CuSO4} electrode? If you have a steel structure 
    that is buried in soil and is located thermodynamically at point A, what
    is the best strategy to have a "corrosion healthy structure"? Name two
    strategies.

    \begin{figure}[h]
        \centering
        \includegraphics[width=0.6\textwidth]{./assets/fig_6.png}
    \end{figure}

\end{enumerate}

\end{document}

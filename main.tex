\input{./src/main.sty}
% Additional SI unit for Fahrenheit
\DeclareSIUnit\fahrenheit{\degree F}

\begin{document}

% Include title page
\input{./src/titlepage.tex}

\section{Problem 1}

Please find the magnitudes missing for each electrical circuit expressed.

\begin{enumerate}[a]
    \item $I_1, I_2, I_\text{total}$ (across the resistance), $V_2, R_\text{total}$?

        \begin{figure}[h]
    \centering
    \includegraphics[width=0.6\textwidth]{./assets/fig_1.png}
\end{figure}

\begin{itemize}
    \item $E_1 = \SI{8}{\volt}$
    \item $E_2 = \SI{4}{\volt}$
    \item $R_1 = \SI{2}{\ohm}$
    \item $R_2 = \SI{4}{\ohm}$
\end{itemize}


\pagebreak

    \item $R_\text{total}, I_1, I_2, I_3, V_1, V_2, V_3$?

        \begin{figure}[h]
    \centering
    \includegraphics[width=0.6\textwidth]{./assets/fig_2.png}
\end{figure}

\begin{itemize}
    \item $E_\text{total} = \SI{5}{\volt}$
    \item $R_2 = \SI{20}{\ohm}$
    \item $R_3 = \SI{30}{\ohm}$
\end{itemize}

\pagebreak

    \item $I_\text{total}, I_1, I_2, I_3, R_\text{total}$?

        \begin{figure}[h]
    \centering
    \includegraphics[width=0.6\textwidth]{./assets/fig_3.png}
\end{figure}

\begin{itemize}
    \item $E_1 = \SI{10}{\volt}$
    \item $E_2 = \SI{20}{\volt}$
    \item $R_1 = \SI{5}{\ohm}$
    \item $R_2 = \SI{10}{\ohm}$
    \item $R_3 = \SI{1}{\ohm}$
    \item $I_4 = \SI{10}{\milli\ampere}$
\end{itemize}

\pagebreak

\item $R_\text{total}, V_\text{total}, I_\text{total}, I_1, I_2, I_3$?

        \begin{figure}[h]
    \centering
    \includegraphics[width=0.6\textwidth]{./assets/fig_4.png}
\end{figure}

\begin{itemize}
    \item $E_1 = \SI{20}{\volt}$
    \item $R_1 = \SI{10}{\ohm}$
    \item $R_2 = \SI{20}{\ohm}$
    \item $R_3 = \SI{5}{\ohm}$
\end{itemize}

\pagebreak

\item Perform the following reference conversions

    \begin{enumerate}
        \item Convert \SI{150}{\milli\volt} SCE to CSE
        \item Convert \SI{-150}{\milli\volt} SCE to CSE
        \item Convert \SI{-1.0}{\volt} CSE to the equivalent Ag/AgCl reference electrode
        \item Convert \SI{-0.75}{\volt} to the equivalent Zn reference electrode
        \item Covert \SI{0.00}{\volt} Zn to the equivalent CSE
        \item Convert \SI{-0.05}{V} Zn to the equivalent Ag/AgCl reference electrode
    \end{enumerate}

\pagebreak

    \item In a laboratory test, the corrosion current density of a steel AISI
        1010 sample is \SI{50}{\milli\ampere\per\centi\meter\squared}, can you 
        calculate the corrosion rate for the alloy?

    Assume the alloy contains the following composition:

    \begin{figure}[h]
        \centering
        \includegraphics[width=0.6\textwidth]{./assets/fig_5.png}
    \end{figure}

    NOTE: Be sure what elements will be anodic or will dissolve in order to 
    make the final calculation of the alloy.

\item If an iron pipeline is immersed in a soil that is $\mathrm{pH}=8$, what 
    electrode potential must be maintained to prevent corrosion of the bare 
    metal versus the \ch{Cu/CuSO4} electrode? If you have a steel structure 
    that is buried in soil and is located thermodynamically at point A, what
    is the best strategy to have a "corrosion healthy structure"? Name two
    strategies.

    \begin{figure}[h]
        \centering
        \includegraphics[width=0.6\textwidth]{./assets/fig_6.png}
    \end{figure}

\end{enumerate}

\end{document}

% Additional SI unit for Fahrenheit
\DeclareSIUnit\fahrenheit{\degree F}

\begin{document}

% Include title page
\input{./src/titlepage.tex}

\section{Problem 1}

Please find the magnitudes missing for each electrical circuit expressed.

\begin{enumerate}[a]
    \item $I_1, I_2, I_\text{total}$ (across the resistance), $V_2, R_\text{total}$?

        \begin{figure}[h]
    \centering
    \includegraphics[width=0.6\textwidth]{./assets/fig_1.png}
\end{figure}

\begin{itemize}
    \item $E_1 = \SI{8}{\volt}$
    \item $E_2 = \SI{4}{\volt}$
    \item $R_1 = \SI{2}{\ohm}$
    \item $R_2 = \SI{4}{\ohm}$
\end{itemize}


\pagebreak

    \item $R_\text{total}, I_1, I_2, I_3, V_1, V_2, V_3$?

        \begin{figure}[h]
    \centering
    \includegraphics[width=0.6\textwidth]{./assets/fig_2.png}
\end{figure}

\begin{itemize}
    \item $E_\text{total} = \SI{5}{\volt}$
    \item $R_2 = \SI{20}{\ohm}$
    \item $R_3 = \SI{30}{\ohm}$
\end{itemize}

\pagebreak

    \item $I_\text{total}, I_1, I_2, I_3, R_\text{total}$?

        \begin{figure}[h]
    \centering
    \includegraphics[width=0.6\textwidth]{./assets/fig_3.png}
\end{figure}

\begin{itemize}
    \item $E_1 = \SI{10}{\volt}$
    \item $E_2 = \SI{20}{\volt}$
    \item $R_1 = \SI{5}{\ohm}$
    \item $R_2 = \SI{10}{\ohm}$
    \item $R_3 = \SI{1}{\ohm}$
    \item $I_4 = \SI{10}{\milli\ampere}$
\end{itemize}

\pagebreak

\item $R_\text{total}, V_\text{total}, I_\text{total}, I_1, I_2, I_3$?

        \begin{figure}[h]
    \centering
    \includegraphics[width=0.6\textwidth]{./assets/fig_4.png}
\end{figure}

\begin{itemize}
    \item $E_1 = \SI{20}{\volt}$
    \item $R_1 = \SI{10}{\ohm}$
    \item $R_2 = \SI{20}{\ohm}$
    \item $R_3 = \SI{5}{\ohm}$
\end{itemize}

\pagebreak

\item Perform the following reference conversions

    \begin{enumerate}
        \item Convert \SI{150}{\milli\volt} SCE to CSE
        \item Convert \SI{-150}{\milli\volt} SCE to CSE
        \item Convert \SI{-1.0}{\volt} CSE to the equivalent Ag/AgCl reference electrode
        \item Convert \SI{-0.75}{\volt} to the equivalent Zn reference electrode
        \item Covert \SI{0.00}{\volt} Zn to the equivalent CSE
        \item Convert \SI{-0.05}{V} Zn to the equivalent Ag/AgCl reference electrode
    \end{enumerate}

\pagebreak

    \item In a laboratory test, the corrosion current density of a steel AISI
        1010 sample is \SI{50}{\milli\ampere\per\centi\meter\squared}, can you 
        calculate the corrosion rate for the alloy?

    Assume the alloy contains the following composition:

    \begin{figure}[h]
        \centering
        \includegraphics[width=0.6\textwidth]{./assets/fig_5.png}
    \end{figure}

    NOTE: Be sure what elements will be anodic or will dissolve in order to 
    make the final calculation of the alloy.

\item If an iron pipeline is immersed in a soil that is $\mathrm{pH}=8$, what 
    electrode potential must be maintained to prevent corrosion of the bare 
    metal versus the \ch{Cu/CuSO4} electrode? If you have a steel structure 
    that is buried in soil and is located thermodynamically at point A, what
    is the best strategy to have a "corrosion healthy structure"? Name two
    strategies.

    \begin{figure}[h]
        \centering
        \includegraphics[width=0.6\textwidth]{./assets/fig_6.png}
    \end{figure}

\end{enumerate}

\end{document}

% Additional SI unit for Fahrenheit
\DeclareSIUnit\fahrenheit{\degree F}

\begin{document}

% Include title page
\input{./src/titlepage.tex}

\section{Problem 1}

Please find the magnitudes missing for each electrical circuit expressed.

\begin{enumerate}[a]
    \item $I_1, I_2, I_\text{total}$ (across the resistance), $V_2, R_\text{total}$?

        \begin{figure}[h]
    \centering
    \includegraphics[width=0.6\textwidth]{./assets/fig_1.png}
\end{figure}

\begin{itemize}
    \item $E_1 = \SI{8}{\volt}$
    \item $E_2 = \SI{4}{\volt}$
    \item $R_1 = \SI{2}{\ohm}$
    \item $R_2 = \SI{4}{\ohm}$
\end{itemize}


\pagebreak

    \item $R_\text{total}, I_1, I_2, I_3, V_1, V_2, V_3$?

        \begin{figure}[h]
    \centering
    \includegraphics[width=0.6\textwidth]{./assets/fig_2.png}
\end{figure}

\begin{itemize}
    \item $E_\text{total} = \SI{5}{\volt}$
    \item $R_2 = \SI{20}{\ohm}$
    \item $R_3 = \SI{30}{\ohm}$
\end{itemize}

\pagebreak

    \item $I_\text{total}, I_1, I_2, I_3, R_\text{total}$?

        \begin{figure}[h]
    \centering
    \includegraphics[width=0.6\textwidth]{./assets/fig_3.png}
\end{figure}

\begin{itemize}
    \item $E_1 = \SI{10}{\volt}$
    \item $E_2 = \SI{20}{\volt}$
    \item $R_1 = \SI{5}{\ohm}$
    \item $R_2 = \SI{10}{\ohm}$
    \item $R_3 = \SI{1}{\ohm}$
    \item $I_4 = \SI{10}{\milli\ampere}$
\end{itemize}

\pagebreak

\item $R_\text{total}, V_\text{total}, I_\text{total}, I_1, I_2, I_3$?

        \begin{figure}[h]
    \centering
    \includegraphics[width=0.6\textwidth]{./assets/fig_4.png}
\end{figure}

\begin{itemize}
    \item $E_1 = \SI{20}{\volt}$
    \item $R_1 = \SI{10}{\ohm}$
    \item $R_2 = \SI{20}{\ohm}$
    \item $R_3 = \SI{5}{\ohm}$
\end{itemize}

\pagebreak

\item Perform the following reference conversions

    \begin{enumerate}
        \item Convert \SI{150}{\milli\volt} SCE to CSE
        \item Convert \SI{-150}{\milli\volt} SCE to CSE
        \item Convert \SI{-1.0}{\volt} CSE to the equivalent Ag/AgCl reference electrode
        \item Convert \SI{-0.75}{\volt} to the equivalent Zn reference electrode
        \item Covert \SI{0.00}{\volt} Zn to the equivalent CSE
        \item Convert \SI{-0.05}{V} Zn to the equivalent Ag/AgCl reference electrode
    \end{enumerate}

\pagebreak

    \item In a laboratory test, the corrosion current density of a steel AISI
        1010 sample is \SI{50}{\milli\ampere\per\centi\meter\squared}, can you 
        calculate the corrosion rate for the alloy?

    Assume the alloy contains the following composition:

    \begin{figure}[h]
        \centering
        \includegraphics[width=0.6\textwidth]{./assets/fig_5.png}
    \end{figure}

    NOTE: Be sure what elements will be anodic or will dissolve in order to 
    make the final calculation of the alloy.

\item If an iron pipeline is immersed in a soil that is $\mathrm{pH}=8$, what 
    electrode potential must be maintained to prevent corrosion of the bare 
    metal versus the \ch{Cu/CuSO4} electrode? If you have a steel structure 
    that is buried in soil and is located thermodynamically at point A, what
    is the best strategy to have a "corrosion healthy structure"? Name two
    strategies.

    \begin{figure}[h]
        \centering
        \includegraphics[width=0.6\textwidth]{./assets/fig_6.png}
    \end{figure}

\end{enumerate}

\end{document}

% Additional SI unit for Fahrenheit
\DeclareSIUnit\fahrenheit{\degree F}
\sisetup{range-phrase = --}
\sisetup{range-units = single}

\begin{document}

% Include title page
\input{./src/titlepage.tex}

Please find the magnitudes missing for each electrical circuit expressed.

\begin{enumerate}[a]
    % Problem 1{{{
  \item $I_1, I_2, I_\text{total}$ (across the resistance), $V_2,
    R_\text{total}$?

    \begin{figure}[h]
      \centering
      \begin{circuitikz}[american]
    % Draw the circuit - starting from bottom left, going clockwise
    \draw (0,0) 
        to[battery1, l=$E_1$] (0,3)
        to[battery1, l=$E_2$] (0,6)
        to[short] (4,6)
        to[R, l=$R_1$] (4,3)
        to[R, l=$R_2$] (4,0)
        to[short] (0,0);
\end{circuitikz}

    \end{figure}

    \begin{itemize}
      \item $E_1 = \SI{8}{\volt}$
      \item $E_2 = \SI{4}{\volt}$
      \item $R_1 = \SI{2}{\ohm}$
      \item $R_2 = \SI{4}{\ohm}$
    \end{itemize}

    \boxedanswer{
      \centering
      \begin{tabular}{|c|c|c|c|}
        \hline
        Component & $\Delta V$ & $I$ & R \\
        \hline
        $E_1$ & \SI{8}{\volt} & \SI{2}{\ampere} & \SI{0}{\ohm} \\
        \hline
        $E_2$ & \SI{4}{\volt} & \SI{2}{\ampere} & \SI{0}{\ohm} \\
        \hline
        $R_1$ & \SI{4}{\volt} & \SI{2}{\ampere} & \SI{2}{\ohm} \\
        \hline
        $R_2$ & \SI{8}{\volt} & \SI{2}{\ampere} & \SI{4}{\ohm} \\
        \hline
      \end{tabular}

      \begin{align*}
        E_\text{total} &= \SI{8}{\volt} + \SI{4}{\volt} \\
        E_\text{total} &= \SI{12}{\volt} \\
        R_\text{total} &= \SI{2}{\ohm} + \SI{4}{\ohm} = \SI{6}{\ohm} \\
        I_\text{total} &= I_1 = I_2 = \frac{V_\text{total}}{R_\text{total}}\\
        I_\text{total} &= \frac{\SI{12}{\volt}}{\SI{6}{\ohm}} \\
        \Aboxed{I_\text{total} = I_1 = I_2 &= \SI{2}{\ampere}} \\
        \Delta V_1 &= I_1R_1 \\
        \Delta V_1 &= (\SI{2}{\ampere})(\SI{2}{\ohm}) \\
        \Aboxed{\Delta V_1 &= \SI{4}{\volt}} \\
        \Delta V_2 &= (\SI{2}{\ampere})(\SI{4}{\ohm}) \\
        \Aboxed{\Delta V_2 &= \SI{8}{\volt}}
      \end{align*}
    }

    \pagebreak% }}}
    % {{{ Problem 2
  \item $R_\text{total}, I_1, I_2, I_3, V_1, V_2, V_3$?

    \begin{figure}[h]
      \centering
      \begin{circuitikz}[american]
  % Draw the circuit - starting from bottom left, going clockwise
  \draw (0,0)
  to[battery1, l=$E$] (0,4)
  to[short] (2,4)
  to[R, l=$R_1$] (4,4)
  to[short] (4,2)
  to[R, l=$R_2$] (4,0)
  to[short] (2,0)
  to[R, l=$R_3$] (0,0);
\end{circuitikz}

    \end{figure}

    \begin{itemize}
      \item $E_\text{total} = \SI{5}{\volt}$
      \item $R_1 = \SI{10}{\ohm}$
      \item $R_2 = \SI{20}{\ohm}$
      \item $R_3 = \SI{30}{\ohm}$
    \end{itemize}

    \boxedanswer{
      \centering
      \begin{tabular}{|c|c|c|c|}
        \hline
        Component & $\Delta V$ & $I$ & R \\
        \hline
        $E$ & \SI{5}{\volt} & \SI{0.0833}{\ampere} & \SI{0}{\ohm} \\
        \hline
        $R_1$ & \SI{0.833}{\volt} & \SI{0.0833}{\ampere} & \SI{10}{\ohm} \\
        \hline
        $R_2$ & \SI{1.667}{\volt} & \SI{0.0833}{\ampere} & \SI{20}{\ohm} \\
        \hline
        $R_3$ & \SI{2.5}{\volt} & \SI{0.0833}{\ampere} & \SI{30}{\ohm} \\
        \hline
      \end{tabular}
      \begin{align*}
        R_\text{total} &= R_1 + R_2 + R_3 \\
        R_\text{total} &= \SI{10}{\ohm} + \SI{20}{\ohm} + \SI{30}{\ohm} \\
        \Aboxed{R_\text{total} &= \SI{60}{\ohm}} \\
        I_\text{total} &= I_1 = I_2 = I_3 =
        \frac{E_\text{total}}{R_\text{total}}\\
        I_\text{total} &= \frac{\SI{5}{\volt}}{\SI{60}{\ohm}} \\
        \Aboxed{I_\text{total} = I_1 = I_2 = I_3 &= \SI{0.0833}{\ampere}} \\
        V_1 &= I_1R_1 \\
        V_1 &= (\SI{0.0833}{\ampere})(\SI{10}{\ohm}) \\
        \Aboxed{V_1 &= \SI{0.833}{\volt}} \\
        V_2 &= I_2R_2 \\
        V_2 &= (\SI{0.0833}{\ampere})(\SI{20}{\ohm}) \\
        \Aboxed{V_2 &= \SI{1.667}{\volt}} \\
        V_3 &= I_3R_3 \\
        V_3 &= (\SI{0.0833}{\ampere})(\SI{30}{\ohm}) \\
        \Aboxed{V_3 &= \SI{2.5}{\volt}}
      \end{align*}
    }

    \pagebreak
    % }}}
    % Problem 3{{{
  \item $I_\text{total}, I_1, I_2, I_3, R_\text{total}$?

    \begin{figure}[h]
      \centering
      \begin{circuitikz}[american]
  % Draw main loop with batteries on left
  \draw (0,0)
  to[battery1, l=$E_1$] (0,3)
  to[battery1, l=$E_2$] (0,6)
  to[short] (6,6);

  \draw (6,0)
  to[short] (0,0);

  % First parallel branch with R_1 and R_2 in series
  \draw (2,6)
  to[R, l=$R_1$] (2,3)
  to[R, l=$R_2$] (2,0);

  % Second parallel branch with R_3
  \draw (4,6)
  to[R, l=$R_3$] (4,0);

  % Third parallel branch with R_4
  \draw (6,6)
  to[R, l=$R_4$] (6,0);
\end{circuitikz}

    \end{figure}

    \begin{itemize}
      \item $E_1 = \SI{10}{\volt}$
      \item $E_2 = \SI{20}{\volt}$
      \item $R_1 = \SI{5}{\ohm}$
      \item $R_2 = \SI{10}{\ohm}$
      \item $R_3 = \SI{1}{\ohm}$
      \item $I_4 = \SI{10}{\milli\ampere}$
    \end{itemize}
    \boxedanswer{
      \centering
      \begin{tabular}{|c|c|c|c|}
        \hline
        Component & $\Delta V$ & $I$ & R \\
        \hline
        $E_1$ & \SI{10}{\volt} & \SI{32}{\ampere} & \SI{0}{\ohm} \\
        \hline
        $E_2$ & \SI{20}{\volt} & \SI{32}{\ampere} & \SI{0}{\ohm} \\
        \hline
        $R_1$ & \SI{10}{\volt} & \SI{2}{\ampere} & \SI{5}{\ohm} \\
        \hline
        $R_2$ & \SI{20}{\volt} & \SI{2}{\ampere} & \SI{10}{\ohm} \\
        \hline
        $R_3$ & \SI{30}{\volt} & \SI{30}{\ampere} & \SI{1}{\ohm} \\
        \hline
        $R_4$ & \SI{30}{\volt} & \SI{10}{\milli\ampere} & \SI{3}{\kilo\ohm} \\
        \hline
      \end{tabular}
      \begin{align*}
        E_\text{total} &= E_1 + E_2 \\
        E_\text{total} &= \SI{10}{\volt} + \SI{20}{\volt} \\
        E_\text{total} &= \SI{30}{\volt} \\
        \text{In parallel: } \Delta V_\text{each branch} &=
        E_\text{total} = \SI{30}{\volt} \\
        R_{1,2} &= R_1 + R_2 = \SI{5}{\ohm} + \SI{10}{\ohm} = \SI{15}{\ohm} \\
        I_1 &= I_2 = \frac{\Delta V}{R_{1,2}} =
        \frac{\SI{30}{\volt}}{\SI{15}{\ohm}} \\
        \Aboxed{I_1 = I_2 &= \SI{2}{\ampere}} \\
        \Aboxed{I_3 &= \frac{\SI{30}{\volt}}{\SI{1}{\ohm}} =
        \SI{30}{\ampere}} \\
        R_4 &= \frac{\Delta V}{I_4} =
        \frac{\SI{30}{\volt}}{\SI{10}{\milli\ampere}} \\
        R_4 &= \SI{3}{\kilo\ohm} \\
        I_\text{total} &= I_1 + I_3 + I_4 \\
        I_\text{total} &= \SI{2}{\ampere} + \SI{30}{\ampere} +
        \SI{10}{\milli\ampere} \\
        \Aboxed{I_\text{total} &= \SI{32.01}{\ampere} \approx
        \SI{32}{\ampere}} \\
        \frac{1}{R_\text{total}} &= \frac{1}{R_{1,2}} + \frac{1}{R_3}
        + \frac{1}{R_4} \\
        \frac{1}{R_\text{total}} &= \frac{1}{\SI{15}{\ohm}} +
        \frac{1}{\SI{1}{\ohm}} + \frac{1}{\SI{3000}{\ohm}} \\
        \frac{1}{R_\text{total}} &= 0.0667 + 1 + 0.000333 =
        \SI{1.067}{\per\ohm} \\
        \Aboxed{R_\text{total} &= \SI{0.938}{\ohm}}
      \end{align*}
    }
    \pagebreak% }}}
    % Problem 4{{{
  \item $R_\text{total}, V_\text{total}, I_\text{total}, I_1, I_2, I_3$?

    \begin{figure}[h]
      \centering
      \begin{circuitikz}[american]
  % Draw the circuit - battery on left, three parallel branches on right
  \draw (0,0)
  to[battery1, l=$E$] (0,4)
  to[short] (2,4)
  to[R, l=$R_1$] (2,0)
  to[short] (0,0);

  \draw (2,4)
  to[short] (4,4)
  to[R, l=$R_2$] (4,0)
  to[short] (2,0);

  \draw (4,4)
  to[short] (6,4)
  to[R, l=$R_3$] (6,0)
  to[short] (4,0);
\end{circuitikz}

    \end{figure}

    \begin{itemize}
      \item $E_1 = \SI{20}{\volt}$
      \item $R_1 = \SI{10}{\ohm}$
      \item $R_2 = \SI{20}{\ohm}$
      \item $R_3 = \SI{5}{\ohm}$
    \end{itemize}

    \pagebreak
    \boxedanswer{
      \centering
      \begin{tabular}{|c|c|c|c|}
        \hline
        Component & $\Delta V$ & $I$ & R \\
        \hline
        $E_1$ & \SI{20}{\volt} & \SI{7}{\ampere} & \SI{0}{\ohm} \\
        \hline
        $R_1$ & \SI{20}{\volt} & \SI{2}{\ampere} & \SI{10}{\ohm} \\
        \hline
        $R_2$ & \SI{20}{\volt} & \SI{1}{\ampere} & \SI{20}{\ohm} \\
        \hline
        $R_3$ & \SI{20}{\volt} & \SI{4}{\ampere} & \SI{5}{\ohm} \\
        \hline
      \end{tabular}
      \begin{align*}
        V_\text{total} &= E_1 \\
        \Aboxed{V_\text{total} &= \SI{20}{\volt}} \\
        I_1 &= \frac{V_\text{total}}{R_1} \\
        I_1 &= \frac{\SI{20}{\volt}}{\SI{10}{\ohm}} \\
        \Aboxed{I_1 &= \SI{2}{\ampere}} \\
        I_2 &= \frac{V_\text{total}}{R_2} \\
        I_2 &= \frac{\SI{20}{\volt}}{\SI{20}{\ohm}} \\
        \Aboxed{I_2 &= \SI{1}{\ampere}} \\
        I_3 &= \frac{V_\text{total}}{R_3} \\
        I_3 &= \frac{\SI{20}{\volt}}{\SI{5}{\ohm}} \\
        \Aboxed{I_3 &= \SI{4}{\ampere}} \\
        I_\text{total} &= I_1 + I_2 + I_3 \\
        I_\text{total} &= \SI{2}{\ampere} + \SI{1}{\ampere} + \SI{4}{\ampere} \\
        \Aboxed{I_\text{total} &= \SI{7}{\ampere}} \\
        \frac{1}{R_\text{total}} &= \frac{1}{R_1} + \frac{1}{R_2} +
        \frac{1}{R_3} \\
        \frac{1}{R_\text{total}} &= \frac{1}{\SI{10}{\ohm}} +
        \frac{1}{\SI{20}{\ohm}} + \frac{1}{\SI{5}{\ohm}} \\
        \frac{1}{R_\text{total}} &= \frac{2}{20} + \frac{1}{20} +
        \frac{4}{20} = \frac{7}{20} \\
        \Aboxed{R_\text{total} &= \SI{2.857}{\ohm}}
      \end{align*}
    }
    % }}}
    % Problem 5{{{
  \item Perform the following reference conversions

    \begin{enumerate}
      \item Convert \SI{150}{\milli\volt} SCE to CSE
      \item Convert \SI{-150}{\milli\volt} SCE to CSE
      \item Convert \SI{-1.0}{\volt} CSE to the equivalent Ag/AgCl
        reference electrode
      \item Convert \SI{-0.75}{\volt} to the equivalent Zn reference electrode
      \item Covert \SI{0.00}{\volt} Zn to the equivalent CSE
      \item Convert \SI{-0.05}{V} Zn to the equivalent Ag/AgCl
        reference electrode
    \end{enumerate}

    \boxedanswer{
      \begin{align*}
        \text{(a) Convert } & \SI{150}{\milli\volt} \text{ SCE to CSE:} \\
        E_{\text{CSE}} &= E_{\text{SCE}} + (V_{\text{SCE}} - V_{\text{CSE}}) \\
        E_{\text{CSE}} &= \SI{150}{\milli\volt} +
        (\SI{241}{\milli\volt} - \SI{314}{\milli\volt}) \\
        E_{\text{CSE}} &= \SI{150}{\milli\volt} - \SI{73}{\milli\volt} \\
        \Aboxed{E_{\text{CSE}} &= \SI{77}{\milli\volt}} \\
        \\
        \text{(b) Convert } & \SI{-150}{\milli\volt} \text{ SCE to CSE:} \\
        E_{\text{CSE}} &= E_{\text{SCE}} + (V_{\text{SCE}} - V_{\text{CSE}}) \\
        E_{\text{CSE}} &= \SI{-150}{\milli\volt} +
        (\SI{241}{\milli\volt} - \SI{314}{\milli\volt}) \\
        E_{\text{CSE}} &= \SI{-150}{\milli\volt} - \SI{73}{\milli\volt} \\
        \Aboxed{E_{\text{CSE}} &= \SI{-223}{\milli\volt}} \\
        \\
        \text{(c) Convert } & \SI{-1.0}{\volt} \text{ CSE to Ag/AgCl:} \\
        E_{\text{Ag/AgCl}} &= E_{\text{CSE}} + (V_{\text{CSE}} -
        V_{\text{Ag/AgCl}}) \\
        E_{\text{Ag/AgCl}} &= \SI{-1.0}{\volt} + (\SI{0.316}{\volt} -
        \SI{0.197}{\volt}) \\
        E_{\text{Ag/AgCl}} &= \SI{-1.0}{\volt} + \SI{0.119}{\volt} \\
        \Aboxed{E_{\text{Ag/AgCl}} &= \SI{-0.881}{\volt}} \\
        \\
        \text{(d) Convert } & \SI{-0.75}{\volt} \text{ CSE to Zn:} \\
        E_{\text{Zn}} &= E_{\text{CSE}} + (V_{\text{CSE}} - V_{\text{Zn}}) \\
        E_{\text{Zn}} &= \SI{-0.75}{\volt} + (\SI{0.316}{\volt} -
        \SI{-0.763}{\volt}) \\
        E_{\text{Zn}} &= \SI{-0.75}{\volt} + \SI{1.079}{\volt} \\
        \Aboxed{E_{\text{Zn}} &= \SI{0.329}{\volt}} \\
        \\
        \text{(e) Convert } & \SI{0.00}{\volt} \text{ Zn to CSE:} \\
        E_{\text{CSE}} &= E_{\text{Zn}} + (V_{\text{Zn}} - V_{\text{CSE}}) \\
        E_{\text{CSE}} &= \SI{0.00}{\volt} + (\SI{-0.763}{\volt} -
        \SI{0.314}{\volt}) \\
        E_{\text{CSE}} &= \SI{0.00}{\volt} - \SI{1.077}{\volt} \\
        \Aboxed{E_{\text{CSE}} &= \SI{-1.077}{\volt}} \\
        \\
        \text{(f) Convert } & \SI{-0.05}{\volt} \text{ Zn to Ag/AgCl:} \\
        E_{\text{Ag/AgCl}} &= E_{\text{Zn}} + (V_{\text{Zn}} -
        V_{\text{Ag/AgCl}}) \\
        E_{\text{Ag/AgCl}} &= \SI{-0.05}{\volt} + (\SI{-0.763}{\volt}
        - \SI{0.197}{\volt}) \\
        E_{\text{Ag/AgCl}} &= \SI{-0.05}{\volt} - \SI{0.960}{\volt} \\
        \Aboxed{E_{\text{Ag/AgCl}} &= \SI{-1.010}{\volt}}
      \end{align*}
    }% }}}
    % Problem 6{{{
    \pagebreak
  \item In a laboratory test, the corrosion current density of a steel AISI
    1010 sample is \SI{50}{\milli\ampere\per\centi\meter\squared}, can you
    calculate the corrosion rate for the alloy?

    Assume the alloy contains the following composition:

    \begin{table}[ht]
      \centering
      \begin{tabular}{|l|c|c|}
        \hline
        SI. No. & Elements  & Content \\
        \hline
        1 & Carbon & \qtyrange{0.080}{0.13}{\percent} \\
        2 & Silicon & $\le$\SI{0.010}{\percent} \\
        3 & Manganese & \qtyrange{0.30}{0.60}{\percent} \\
        4 & Sulphur & $\le$\SI{0.050}{\percent} \\
        5 & Phosphorous & $\le$\SI{0.040}{\percent} \\
        6 & Iron & \qtyrange{99.08}{99.52}{\percent} \\
        \hline
      \end{tabular}
    \end{table}

    NOTE: Be sure what elements will be anodic or will dissolve in order to
    make the final calculation of the alloy.
    \boxedanswer{
      \begin{align*}
        \text{Mass fractions:} \quad f_\text{Fe} &= 0.993, \quad
        f_\text{Mn} = 0.0045 \\
        \text{Equivalent weights:} \quad \text{EW}_\text{Fe} &=
        \frac{55.845}{2} = \SI{27.923}{\gram\per\equiv} \\
        \text{EW}_\text{Mn} &= \frac{54.938}{2} =
        \SI{27.469}{\gram\per\equiv} \\
        \frac{1}{\text{EW}_\text{alloy}} &=
        \frac{f_\text{Fe}}{\text{EW}_\text{Fe}} +
        \frac{f_\text{Mn}}{\text{EW}_\text{Mn}} \\
        \frac{1}{\text{EW}_\text{alloy}} &= \frac{0.993}{27.923} +
        \frac{0.0045}{27.469} \\
        \frac{1}{\text{EW}_\text{alloy}} &= 0.03556 + 0.00016 = 0.03572 \\
        \Aboxed{\text{EW}_\text{alloy} &= \SI{27.997}{\gram\per\equiv}} \\
        \rho_\text{steel} &= \SI{7.87}{\gram\per\centi\meter\cubed} \\
        i_\text{corr} &= \SI{50}{\milli\ampere\per\centi\meter\squared} \\
        K &=
        \SI{3.27e-3}{\milli\meter\per\year\per\milli\ampere\per\centi\meter\squared}
        \\
        \text{CR} &= K \times \frac{i_\text{corr} \times
        \text{EW}_\text{alloy}}{\rho} \\
        \text{CR} &= 3.27 \times 10^{-3} \times \frac{50 \times 27.997}{7.87} \\
        \text{CR} &= 3.27 \times 10^{-3} \times 177.87 \\
        \Aboxed{\text{CR} &= \SI{0.582}{\milli\meter\per\year}}
      \end{align*}
      \pagebreak
    }% }}}
    % Problem 7{{{

  \item If an iron pipeline is immersed in a soil that is $\mathrm{pH}=8$, what
    electrode potential must be maintained to prevent corrosion of the bare
    metal versus the \ch{Cu/CuSO4} electrode? If you have a steel structure
    that is buried in soil and is located thermodynamically at point A, what
    is the best strategy to have a "corrosion healthy structure"? Name two
    strategies.

    \boxedanswer{
      The potential must be maintained below \SI{-0.5}{\volt} vs \ch{Cu/CuSO4}
      , assuming the reference for the provided Pourbaix diagram is versus
      \ch{Cu/CuSO4}. \par

      Two strategies to prevent corrosion would be:

      \begin{enumerate}
        \item Apply a potential of \SI{-0.2}{\volt}
        \item Decrease the pH to below 2
      \end{enumerate}
    }

    \begin{figure}[ht]
      \centering
      \includegraphics[width=0.6\textwidth]{./assets/fig_6_annotated.png}
    \end{figure}

\end{enumerate}% }}}

\end{document}
